\documentclass[12pt]{article}

\usepackage[T1]{fontenc}
\usepackage[french]{babel}
\usepackage{amsmath}
\usepackage{amsthm}
\usepackage{amssymb}
\usepackage[colorlinks=true, linkcolor=blue, citecolor=red]{hyperref}
\usepackage{graphicx}
\usepackage[left=3cm,right=3cm,top=3cm,bottom=3cm]{geometry}
\usepackage{enumitem}
\usepackage{array}
\usepackage{stmaryrd}
\usepackage{tkz-euclide}
\usepackage{tikz}
	\usetikzlibrary{decorations.pathreplacing,decorations.markings, intersections, calc, angles, arrows}
\usepackage{epsfig}


%________________________________________________________________

\usepackage{amsthm}

\newtheorem{theo}{Théorème}
\newtheorem{prop}{Proposition}
\newtheorem{defi}{Définition}
\newtheorem{expl}{Exemple}
\newtheorem{coro}{Corolaire}


% text to be compile

\begin{document}

\author{Sedjro}
\title{To be modified title}





%-----------------------------------------------------------------------------------------------

\section{Introduction}
Souvent SEDJRO Friedly un problème concret fourni par l'analyse, la mécanique, la physique, la biologie , l'économie ,... se ramène à la résolution d'équations ou de systèmes différentielle. C'est dans cette otique que j'ai décide de vous faire un bref résumé de la résolution de ces gens de problème. 
\section{}
%----------------------------------------------

\subsection{Équation différentiel linéaire du premier ordre}


\begin{defi}
On appelle equation linéaire du premier ordre une equation de la forme:
\begin{equation}
\forall x \in \mathbb{R}, y'(x)+a(x)y(x)=f(x)
\end{equation}
ou $a$ et $f$ sont deux fonctions données, continues sur $I$ un intervalle non vide de  $\mathbb{R}$.
La fonction $a$ s'appelle de coefficient et le fonction $f$ le second membre de l'équation $(E)$.
\end{defi}

\begin{expl}
L'équation suivante est une equation différentiel linéaire du premier ordre:
\begin{equation}
\forall x\in \mathbb{R}, y'(x)+2xy(x)=x
\end{equation}

\end{expl}

\begin{defi}Définition
On appelle équation homogène associée à $(E)$ l'équation $(E_{H})$ obtenue en remplaçant $f$ par $0$ dans $(E)$:
\begin{equation}
\forall x \in I, y'(x)+a(x)y(x)=0
\end{equation}
\end{defi}


\begin{prop}
Si $y_{1}$ et $y_{2}$ sont deux solutions de l'équation $(E)$, alors $y_{1}-y_{2}$ est une solution de $(E_{H})$.
\end{prop}


\begin{theo}
Soit $S_{H}$ l'ensemble des solutions de $E_{H}$, l'équation homogène associée à $(E)$ et soit $yp$ une solution particulière de $(E)$.
Alors l'ensemble $S$ des solutions de $(E)$ est défini par: 
$$ S=y_{p}+y_{h},$$\quad $y_{h} \in S_{h}$
\end{theo}


\begin{coro}
Soit $A$ une primitive quelconque de $a$ sur $I$ et soit $y_{p}$ une solution particulière de $(E)$. Alors, nous avons 
\begin{equation}
 S = \{y_{p} +y_{h},\quad y_{h} \in S_{h}\}
\end{equation} 
\end{coro}



\subsection{Recherche de solution particulière par variation de la constante}

Nous allons déterminer une solution particulière $Y_{p}$ de la forme $ \forall x\in I,$ \qquad$ y_{p}(x)= v(x)y_{H}(x)$
ou $y_{H}(x)$ est une solutions de $(E_{h}$, l'équation homogène associée à $(E)$ est $v$ est une fonction dérivable du $I$ à déterminer.\newline
En effet, si $y_{p}$ est solutions de $(E)$ alors $\forall x \in I$
\begin{align}
y'_{p}(x)+ a(x)y_{p}(x)=f(x)
\\v'(x)y_{H}(x)+ v(x)y'_{H}(x) + a(x)v(x)y_{H}(x) = f(x) 
\\v'(x)y_{H}(x)+ v(x)( y'_{H}(x)+ a(x)y_{H}(x))=f(x)
\end{align}
d'où puisque $y_{H}$ ne s'annule pas sur $\mathbb{I}$:
\begin{equation}
\forall x \in I,  v'(x)=\frac{f(x)}{y_{H}(x)}
\end{equation}

Nous sommes ainsi ramenés a un nouveau à la recherche d'une primitive .
Dans la pratique, les solutions de l'équation homogène associés sont de la forme $x\longrightarrow C \exp(x) $


\eqref{1}





















\bibliographystyle{abbrv}
\bibliography{biblio}



\end{document}