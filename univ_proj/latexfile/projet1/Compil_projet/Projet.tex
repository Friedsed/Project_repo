\documentclass[11pt]{report}

% ---------- Packages ----------
\usepackage[utf8]{inputenc}
\usepackage[T1]{fontenc}
\usepackage[english]{babel}
\usepackage{lmodern}
\usepackage{geometry}
\geometry{margin=1in}
\usepackage{setspace}
\onehalfspacing
\usepackage{graphicx}
\usepackage{amsmath,amssymb}
\usepackage{listings}
\usepackage{xcolor}
\usepackage{hyperref}
\usepackage{graphicx}

% ---------- Code listing style ----------
\lstdefinestyle{csstyle}{
	language=Python,
	basicstyle=\ttfamily\small,
	numbers=left,
	numberstyle=\tiny,
	stepnumber=1,
	showstringspaces=false,
	keywordstyle=\color{blue},
	commentstyle=\color{gray},
	stringstyle=\color{red},
	frame=single,
	breaklines=true
}
\lstset{style=csstyle}

% ---------- Meta data ----------
\title{Project Title in Computer Science}
\author{Your Name \\
	Department of Computer Science \\
	Your University}
\date{\today}

\begin{document}
	
	
	$/home/friedly/Documents/General_git_hub_file/all_latex_projet/info_project/Images/contour3.png$
	
	\pagenumbering{roman}
	\maketitle
	
	\begin{abstract}
		This document describes the structure of a typical computer science
		project report written in \LaTeX. Replace this text with your own abstract.
	\end{abstract}
	
	\tableofcontents
	\listoffigures
	\listoftables
	
	\clearpage
	\pagenumbering{arabic}
	

	% ---------- 2. Background ----------
	\chapter{Background}
	Describe the theoretical background, related work, and key concepts.
	
	\section{Related Work}
	Summarize existing approaches and why they are not sufficient.
	
	\section{Theoretical Concepts}
	Introduce the main algorithms, data structures, or models.
	
	% ---------- 3. System Design ----------
	\chapter{System Design}
	Explain the architecture and design decisions of your system.
	
	\section{Architecture Overview}
	Describe the system components and how they interact.
	
	\section{Data Structures and Algorithms}
	Detail the main data structures and algorithms used.
	
	% ---------- 4. Implementation ----------
	\chapter{Implementation}
	Describe how the system was implemented, including tools and languages.
	
	\section{Technologies Used}
	List programming languages, frameworks, libraries, and tools.
	
	\section{Example Code}
	Example of including source code (Python here):
	
	\begin{lstlisting}[caption={Sample function}, label={lst:sample-func}]
		def factorial(n: int) -> int:
		"""Compute n! recursively."""
		if n <= 1:
		return 1
		return n * factorial(n - 1)
	\end{lstlisting}
	
	
	\begin{figure}[h!]
		\centering
%		\includegraphics[width=0.7\textwidth]{/home/friedly/Documents/General_git_hub_file/all_latex_projet/info_project/Images/contour3.png}
		\caption{affichage de contour}
		\label{fig:Contours}
	\end{figure}
	
	
	% ---------- 5. Results and Evaluation ----------
	\chapter{Results and Evaluation}
	Present experimental setup, datasets, metrics, and results.
	
	\section{Experimental Setup}
	Describe hardware, software versions, and datasets used.
	
	\section{Results}
	Summarize your results, including tables and figures if needed.
	
	\section{Discussion}
	Interpret the results and discuss limitations.
	
	% ---------- 6. Conclusion ----------
	\chapter{Conclusion}
	Recap the work, main contributions, and possible future improvements.
	
	\section{Future Work}
	List possible extensions or improvements of the project.
	
	% ---------- Bibliography ----------
	\begin{thebibliography}{9}
		
		\bibitem{lamport}
		Leslie Lamport.
		\textit{\LaTeX: A Document Preparation System}.
		Addison-Wesley, 2nd edition, 1994.
		
	\end{thebibliography}
	
\end{document}
