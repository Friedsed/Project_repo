\documentclass[11pt,a4paper]{article}

% =========================
% PACKAGES
% =========================
\usepackage[utf8]{inputenc}      % Encodage UTF-8
\usepackage[T1]{fontenc}         % Encodage des fontes
\usepackage[french]{babel}       % Typographie française [web:7][web:9]
\usepackage{amsmath, amssymb}    % Maths avancées [web:11]
\usepackage{physics}             % Notation vectorielle, dérivées, etc.
\usepackage{graphicx}            % Insertion d'images
\usepackage{siunitx}             % Unités SI
\usepackage{geometry}            % Marges
\usepackage{hyperref}            % Liens cliquables (TOC, références, URLs)
\usepackage{float}               % Contrôle des flottants (H, h, etc.)
\usepackage{caption}             % Légendes
\usepackage{subcaption}          % Sous-figures
\usepackage[absolute,overlay]{textpos} % pour le placement absolu

% Réglages de mise en page
\geometry{margin=2.5cm}
\setlength{\parskip}{1em}

% Répertoire des figures
\graphicspath{{../Images/}}
\usepackage{media9}


% =========================
% INFORMATIONS
% =========================
\title{\textbf{Battery container report}}
\author{
	Toulouse racing \\ \\
	Powertrain division \\ \\
	\\ \\ \\ \\ \\ \\ \\ \\ \\ \\ 
	\textbf{WOLI Sedjro Friedly }
	\\ \\ \\ \\ \\ \\ \\ \\ \\ \\  \\ \\ \\ \\ \\ \\ \\ \\ \\ \\ 
}
\date{Année universitaire 2025--2026}

\begin{document}
		
	
	\maketitle
	\thispagestyle{empty}
	\newpage
	
	\tableofcontents
	\newpage
	
	
	\section{Battery Container Simulation on Ansys}
	
	This section presents the finite element analysis (FEA) of a battery container carried out using Ansys Mechanical. The objective is to evaluate the structural response of the container under combined pressure and acceleration loads representative of in-service conditions.
		
	
		\subsection{Geometry}
		
		The initial geometry consisted of a container with two lateral pipes on each side, which significantly increased the geometrical complexity and led to convergence difficulties during the simulation. 
		To obtain a stable and computationally efficient solution, the model was simplified to a single-chamber container while preserving the main load paths and boundary conditions relevant for the structural assessment.
		
		This simplification reduces the number of elements and improves mesh quality, while still providing a sufficiently accurate representation of the global stiffness and deformation behavior of the structure.
		
		\begin{figure}[H]
			\centering
			\includegraphics[width=0.7\textwidth]{geometry0.png}
			\caption{Geometry used  }
			\label{fig: Geometry }
		\end{figure}
		
		
		\subsection{Material characteristic}
		
		
		Aluminum is selected as the simulation material to match the target container design, providing realistic stiffness and strength predictions under the applied loading conditions.
		
		\begin{figure}[H]
			\centering
			\includegraphics[width=0.7\textwidth]{aluminuim.png}
			\caption{Aluminum  }
			\label{fig: Material}
		\end{figure}
		
		\subsection{Loading and Boundary Conditions}
		
			\subsubsection{Applied Pressure}
			
				A uniform pressure of \(p = 1000~\text{Pa}\) is applied on the bottom surface of the container to represent the effect of the internal pressure acting on the structure due to the weight.
				
				\begin{figure}[H]
					\centering
					\includegraphics[width=0.7\textwidth]{pressure_on_bottom_surface.png}
					\caption{Pressure on the bottom surface}
					\label{fig: Pressure force}
				\end{figure}
			
			\subsubsection{Fixed Supports}
			
				To simulate the real-world constraints of the container mounted in a larger assembly, fixed supports (zero-displacement boundary conditions) are applied along four bottom lines of the structure.
				These fixed supports prevent any translational motion at the constrained locations.
				
				\begin{figure}[H]
					\centering
					\includegraphics[width=0.7\textwidth]{fixed_support.png}
					\caption{ Fixed supports }
					\label{fig; Fixed supprot}
				\end{figure}
				
				
				
			\subsubsection{Acceleration Load}
			
				In addition to the pressure loading, a total acceleration field is applied according to rule-based requirements:
				\begin{itemize}
					\item \(40g\) in the longitudinal direction (\(x\)),
					\item \(40g\) in the lateral direction (\(y\)),
					\item \(20g\) in the vertical direction (\(z\)).
				\end{itemize}
		
				
				These components are combined to obtain a resultant acceleration vector of the form \[ \vec{a} = 40g\,\vec{x} + 40g\,\vec{y} + 20g\,\vec{z} \] . 
			
				
				\begin{figure}[H]
					\centering
					\includegraphics[width=0.7\textwidth]{acceleration1.png}
					\caption{Acceleration on the container }
					\label{fig: Acceleration}
				\end{figure}
			
			\subsection{Meshing Strategy and Numerical Constraints}
			
				\subsubsection{Final Mesh Settings}
				
					A structured meshing strategy is employed with a refined element size on the bottom surface, where stresses and deformations are expected to be critical.
					The final mesh uses a bottom surface element sizing of \(0.01~\text{m}\).
					
					\begin{figure}[H]
						\centering
						\includegraphics[width=0.7\textwidth]{meshing1.png}
						\caption{Meshing result }
						\label{fig: meshing result }
					\end{figure}
					
				
				
			
			\subsection{Total Deformation}
			
				The total deformation plot represents the magnitude of displacement \(|\vec{u}|\) experienced by each point of the container under the combined effect of the 1000~Pa pressure and the applied acceleration field. 
		
				
				These results help determine whether the container satisfies functional requirements, such as maximum allowable deflection, and provide guidance for potential design improvements.
				
				\begin{figure}[H]
					\centering
					\includegraphics[width=0.7\textwidth]{total_deformation.png}
					\caption{Total deformation }
					\label{fig: Total deformation }
				\end{figure}
				
				Red zones max deformation  $ 3.5 $ mm  occur at the top front corner and upper walls, driven by bending under inertial forces.
				
				Blue or green areas near fixed lines show near-zero displacement, confirming proper boundary conditions.
				
				
			\subsection{Directional Deformation}
		
				Directional deformation plots are used to analyze displacements along specific axes (e.g., \(x\)-, \(y\)-, or \(z\)-direction), which is useful for evaluating relative motion in directions that are critical for assembly interfaces or clearance requirements. 
				
				\begin{figure}[H]
					\centering
					\includegraphics[width=0.7\textwidth]{directionnal_defromation.png}
					\caption{Directionnal deformation }
					\label{fig:directional deformation}
				\end{figure}
				
			\subsection{Equivalent (von Mises) Stress}
		
				The equivalent von Mises stress field is computed to evaluate the structural integrity of the container under the combined loading scenario.
				Regions showing the highest von Mises stress are typically located near load introduction points or geometric discontinuities, where stress concentrations develop. 
				
				\begin{figure}[H]
					\centering
					\includegraphics[width=0.7\textwidth]{equivqlent_stress0.png}
					\caption{Equivalent Von Mises Stress}
					\label{fig: Stress}
				\end{figure}
				
				
				\subsection{Some video for understanding }
				
					You have some video and other picture to strengthen your understanding of the simulation result.
					
				
		
	% ==================================================
	\section{Conclusion}
	% =================================================
			
		\textbf{Potential Improvements}
		
		Increase wall thickness or add ribs at red regions to reduce peak deformation . 
		
		
	
		\textbf{I stay open to any suggestion of correction}
		
		\vspace{1cm}
		\begin{center}
			\textcolor[RGB]{34,139,34}{\textbf{Thank you for your reading.}}
		\end{center}
		
		
		
\end{document}
