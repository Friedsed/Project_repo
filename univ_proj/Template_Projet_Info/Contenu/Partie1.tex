\clearpage
\usepackage{CRinfo} %Import du fichier de style (format du document)
\graphicspath{{../Outputs/}}
\section{Titre de votre partie 1}




	\subsection{Titre de votre sous-partie 1}
	
		\begin{figure}[H]
			\centering
			\includegraphics[width=0.7\textwidth]{plot_trajectory_m1.png}
			\caption{Example of a projectile trajectory \(z(x)\) for Model~1.}
			\label{fig:plot_trajectory}
		\end{figure}
	
	
	
	
	\subsubsection{Titre de votre sous-sous-partie 1}
	
	\lipsum[1]
	
		\begin{figure}[H]
			\centering
			\includegraphics[width=0.7\textwidth]{trajectories_m1.png}
			\caption{Example of a projectile trajectory \(z(x)\) for Model~1.}
			\label{fig:plot_trajectory}
		\end{figure}
	
	
	
	
	\subsubsection{Titre de votre sous-sous-partie 2}
	\lipsum[2]
		\begin{figure}[H]
			\centering
			\includegraphics[width=0.7\textwidth]{plot_components_m1.png}
			\caption{Example of a projectile trajectory \(z(x)\) for Model~1.}
			\label{fig:plot_trajectory}
		\end{figure}
		
		
		
	\subsection{Titre de votre sous-partie 2}
		\begin{figure}[H]
			\centering
			\includegraphics[width=0.7\textwidth]{fonction_alpha_m1.png}
			\caption{Example of a projectile trajectory \(z(x)\) for Model~1.}
			\label{fig:plot_trajectory}
		\end{figure}
	\subsubsection{Titre de votre sous-sous-partie 1}
	\lipsum[3]
	
	\subsubsection{Titre de votre sous-sous-partie 2}
	\lipsum[4]
