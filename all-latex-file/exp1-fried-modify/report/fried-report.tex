% =============================================================================
% DÉBUT PRÉAMBULE - TEMPLATE THÈSE/RAPPORT PROFESSIONNEL
% =============================================================================

\documentclass[12pt, a4paper, final, twoside]{report}
% 12pt = taille police principale
% a4paper = format papier A4 standard
% final = supprime messages debug/warnings
% twoside = mise en page livre (marges alternées verso/recto)

\newcommand{\isPrintVersion}{false}						
% Variable de contrôle : false = liens COLORÉS (version PDF écran)
% true = tous liens NOIRS (version impression papier)

\usepackage[utf8]{inputenc}
% Encodage UTF-8 : support accents français/allemands/espagnols/etc.

%%%%%%%%%%%%%%%%%%%%%%%%%%%%%%%%%%%%%%%%%%%%%%%%%%%%%%%%%%%%%%%%%%%%%%%%%%%%%%%%
%%%%%%%%%%%%%%%%%%%%%%%%%%%%%%% DONNÉES THÈSE/RAPPORT %%%%%%%%%%%%%%%%%%%%%%%%%%%
%%%%%%%%%%%%%%%%%%%%%%%%%%%%%%%%%%%%%%%%%%%%%%%%%%%%%%%%%%%%%%%%%%%%%%%%%%%%%%%%
\newcommand{\thesisTitle}{Etude de projectile balistique}
% Titre principal du document (utilisé partout : page titre, PDF metadata...)
\newcommand{\authorName}{Friedly WOLI}
% Nom auteur principal
\newcommand{\thesisDate}{$19^{th}$ Januar 2038}
% Date thèse : $19^{th}$ = exposant "th" pour "19th"

\newcommand{\city}{Toulouse}
% Ville du projet/université
%%%%% Éléments page de garde détaillée
\newcommand{\reportType}{Master Thesis}
% Type document : "Project Report", "Master Thesis", "PhD Dissertation"...
\newcommand{\courseOfStudies}{ Mechanical Engineering }
% Filière/formation : "Mechanical Engineering", "Informatique"...
\newcommand{\university}{Universite de Toulouse }
% Nom université complète
\newcommand{\timeOfProject}{01/1970 -- 01/2038}
% Période stage/projet : format MM/YYYY -- MM/YYYY
\newcommand{\studentId}{1234567}
% Numéro étudiant
\newcommand{\course}{XYZ1234}
% Code matière/cours
\newcommand{\company}{MyCompany}
% Entreprise stage (si applicable)
\newcommand{\supervisorName}{William Tanner}
% Tuteur principal
\newcommand{\secondSupervisorName}{Jane Doe}
% Deuxième correcteur (optionnel)
%%%%% Labels interface (multilingue anglais/allemand)
\newcommand{\byTitle}{by}
\newcommand{\courseOfStudiesTitle}{Course of Studies}
\newcommand{\timeOfProjectTitle}{Time of Project}
\newcommand{\studentIdTitle}{Student Number}
\newcommand{\courseTitle}{Course}
\newcommand{\companyTitle}{Company}
\newcommand{\supervisorTitle}{Supervisor}
\newcommand{\secondSupervisorTitle}{Reviewer}
% Permet traduction facile byTitle → "par", "von", etc.
\newcommand{\universityLogo}{example-image-16x9.pdf}
% Fichier logo université (remplacer par vrai logo)
\newcommand{\companyLogo}{example-image-16x9.pdf}
% Fichier logo entreprise (remplacer par vrai logo)



%%%%% Titres des listes automatiques (anglais/allemand)
\newcommand{\ListOfFiguresTitle}{List of Figures}
\newcommand{\ListOfAbbreviationsTitle}{List of Abbreviations}
\newcommand{\ListOfTablesTitle}{List of Tables}
\newcommand{\ListOfListingsTitle}{List of Listings}
% Personnalise titres : "Liste des figures", "Glossaire", etc.
%%%%% Commandes globales titre/auteur (pour \maketitle)
\title{\thesisTitle}
\author{\authorName}


%%%%%%%%%%%%%%%%%%%%%%%%%%%%%%%%%%%%%%%%%%%%%%%%%%%%%%%%%%%%%%%%%%%%%%%%%%%%%%%%
%%%%%%%%%%%%%%%%%%%%%%%%%%%%%% PAQUETS FONDAMENTAUX %%%%%%%%%%%%%%%%%%%%%%%%%%%%%
%%%%%%%%%%%%%%%%%%%%%%%%%%%%%%%%%%%%%%%%%%%%%%%%%%%%%%%%%%%%%%%%%%%%%%%%%%%%%%%%
\usepackage[left=3.5cm,right=2.5cm,top=3cm,bottom=2.5cm]{geometry}
% Marges précises : 
% 3.5cm GAUCHE = reliure + notes marge
% 2.5cm DROIT = texte confortable
% 3cm HAUT = en-têtes
% 2.5cm BAS = pieds page

\usepackage[francais,english]{babel}
% Langues : francais (priorité) + anglais
% Traduit automatiquement : "Chapter", "Figure", "Bibliography"...

\usepackage{amsmath}
\usepackage{amsfonts}
\usepackage{amssymb}
% Paquets maths AMS : équations complexes, symboles spéciaux (∇, ∑, ∫...)
\usepackage{enumitem}
% Listes personnalisées : espacement, puces, numérotation avancée
\usepackage{float}
% Positionnement figures : [H] = ICI et nulle part ailleurs
\usepackage{fancyhdr}
% En-têtes/pieds de page COMPLETEMENT personnalisables
\usepackage[bottom]{footmisc}
% Notes de bas de page ANCRE en bas (pas milieu)
\usepackage[printonlyused]{acronym}
% Abréviations intelligentes : première occurrence = "CFD (Computational Fluid Dynamics)"
\usepackage{appendix}
% Annexes : A, B, C... (pas continuation chapitres)
\usepackage{pdfpages}
% Insertion pages PDF externes (plans CAO, rapports joints...)
\usepackage{ifthen}
% Conditions logiques : \ifthenelse{condition}{vrai}{faux}
\usepackage{csquotes}
% Guillemets typographiques «français» "anglais" „allemand“
\usepackage{emptypage}
% Pages vides SANS numéro (après chapitres pairs)

%%%%%%%%%%%%%%%%%%%%%%%%%%%%%%%%%%%%%%%%%%%%%%%%%%%%%%%%%%%%%%%%%%%%%%%%%%%%%%%%
%%%%%%%%%%%%%%%%%%%%%%%%%%%%%% COULEURS CMYK (IMPRESSION PRO) %%%%%%%%%%%%%%%%%%
%%%%%%%%%%%%%%%%%%%%%%%%%%%%%%%%%%%%%%%%%%%%%%%%%%%%%%%%%%%%%%%%%%%%%%%%%%%%%%%%
\definecolor{darkBlue}{cmyk}{0.9592, 0.4592, 0, 0.6157} 
% Bleu foncé liens (#043562) - Mode CMYK pour imprimerie
\definecolor{lightBlue}{cmyk}{0.5, 0.25, 0, 0.2} 		
% Bleu clair cadres (#6699cc)
\definecolor{darkGrey}{cmyk}{0, 0, 0, 0.9333} 			
% Gris foncé texte code (#111111)
\definecolor{lightGrey}{cmyk}{0.0164, 0.0082, 0, 0.0431}
% Gris très clair fond code (#F0F2F4)
\definecolor{magenta}{cmyk}{0, 1, 0.4972, 0.3059}		
% Magenta keywords code (#B10059)

%%%%%%%%%%%%%%%%%%%%%%%%%%%%%%%%%%%%%%%%%%%%%%%%%%%%%%%%%%%%%%%%%%%%%%%%%%%%%%%%
%%%%%%%%%%%%%%%%%%%%%%%%%%%%%% NOUVELLES COMMANDES (RACCOURCIS) %%%%%%%%%%%%%%%%%
%%%%%%%%%%%%%%%%%%%%%%%%%%%%%%%%%%%%%%%%%%%%%%%%%%%%%%%%%%%%%%%%%%%%%%%%%%%%%%%%
%%%%% Légendes avec source
\usepackage{caption}									
% Personnalisation légendes figures/tableaux

\captionsetup{justification=centering}
% TOUTES légendes CENTRÉES

\newcommand{\captionsource}[2][]{
	\ifthenelse{\equal{#1}{}}					% Si 1er argument VIDE
	{\small\textit{Source: #2}} 				% → Source simple
	{\small\textit{Source: #2, \href{#1}{#1}}} % → Source + LIEN cliquable
}
% Usage : \captionsource{https://...}{Wikipedia}
%%%%% Boîtes théorèmes/définitions
\usepackage{amsthm} 					
\newtheorem{define}{Definition}
% Crée boîtes numérotées : "Définition 1", "Définition 2"...
%%%%% Code inline texte
\newcommand{\code}[1]{{\ttfamily#1}}
% `monCode()` → police monospace
%%%%%%%%%%%%%%%%%%%%%%%%%%%%%%%%%%%%%%%%%%%%%%%%%%%%%%%%%%%%%%%%%%%%%%%%%%%%%%%%
%%%%%%%%%%%%%%%%%%%%%%%%%%%%%% BIBLIOGRAPHIE NUMÉRIQUE %%%%%%%%%%%%%%%%%%%%%%%%%%
%%%%%%%%%%%%%%%%%%%%%%%%%%%%%%%%%%%%%%%%%%%%%%%%%%%%%%%%%%%%%%%%%%%%%%%%%%%%%%%%
\usepackage[backend=bibtex,style=numeric,citestyle=numeric]{biblatex}
% Biblatex moderne : [1], [2], [3] numéroté
% backend=bibtex (pas biber pour compatibilité)

\bibliography{bibliography} 		
% Fichier sources : bibliography.bib

% COUPURE URLS BIBLIO (CRUCIAL pour longues URLs)
\usepackage{url}							% URLs intelligentes
\usepackage{breakurl}						% Coupures forcées
\def\UrlBreaks{\do\/\do-}					% AUTORISE coupe après / et -
\setcounter{biburlnumpenalty}{1000}			% Coupe FACILE après CHIFFRES
\setcounter{biburlucpenalty}{1000}			% Coupe FACILE après MAJUSCULES
\setcounter{biburllcpenalty}{1000}			% Coupe FACILE après minuscules

%%%%%%%%%%%%%%%%%%%%%%%%%%%%%%%%%%%%%%%%%%%%%%%%%%%%%%%%%%%%%%%%%%%%%%%%%%%%%%%%
%%%%%%%%%%%%%%%%%%%%%%%%%%%%%%% HYPERLIENS PDF PRO %%%%%%%%%%%%%%%%%%%%%%%%%%%%%%
%%%%%%%%%%%%%%%%%%%%%%%%%%%%%%%%%%%%%%%%%%%%%%%%%%%%%%%%%%%%%%%%%%%%%%%%%%%%%%%%
\usepackage[pdftitle={\thesisTitle}, pdfauthor={\authorName},pdfsubject={\thesisTitle}, pdfcreator={pdflatex}, pdfpagemode=UseOutlines, pdfdisplaydoctitle=true, pdflang={en}, breaklinks]{hyperref}
% Métadonnées PDF complètes + table matières cliquable + liens cassables

% LIENS selon version impression/écran
\ifthenelse{\equal{\isPrintVersion}{true}}
{ % ===== VERSION IMPRESSION (noir pur) =====
	\hypersetup{
		hidelinks=true,							% Liens invisibles (cadres noirs)
		colorlinks=false, 
		linktocpage=true, 						% N° pages TOC cliquables
		bookmarksnumbered=true 					% Chapitre 1.1 dans PDF TOC
	}
}{ % ===== VERSION ÉCRAN (couleurs) =====
	\hypersetup{
		colorlinks=true, 						% Liens colorés
		linkcolor=darkBlue,						% Liens internes = bleu foncé
		citecolor=darkBlue,						% Citations [1] = bleu
		filecolor=darkBlue,						% Fichiers joints
		menucolor=darkBlue,
		urlcolor=darkBlue,						% URLs externes
		linktocpage=true,
		bookmarksnumbered=true
	}
}

%%%%%%%%%%%%%%%%%%%%%%%%%%%%%%%%%%%%%%%%%%%%%%%%%%%%%%%%%%%%%%%%%%%%%%%%%%%%%%%%
%%%%%%%%%%%%%%%%%%%%%%%%%%%%%%% TYPOGRAPHIE PROFESSIONNELLE %%%%%%%%%%%%%%%%%%%%%
%%%%%%%%%%%%%%%%%%%%%%%%%%%%%%%%%%%%%%%%%%%%%%%%%%%%%%%%%%%%%%%%%%%%%%%%%%%%%%%%

\setlength{\parskip}{8pt}							% ESPACE 8pt APRÈS paragraphe
\setlength{\parindent}{0pt}							% SANS alinéa (avec parskip)
\renewcommand{\baselinestretch}{1.2}\normalsize		% Interlignes 120% + taille normale

% Évite lignes isolées (veuves/orphelines)
\clubpenalty = 10000								% Jamais ligne orpheline début page
\widowpenalty = 10000 							% Jamais ligne veuve fin page
\displaywidowpenalty=10000						% Équations aussi

% TITRES CHAPITRES personnalisés
\usepackage{titlesec}								% Contrôle total titres
\newcommand{\hsp}{\hspace{20pt}}					% Espace 20pt raccourci

\titleformat{\chapter}[block]{\Huge\bfseries}{\thechapter \hsp {|}\hsp}{0pt}{\Huge\bfseries}
% Chapitre : "1 | TITRE" format GÉANT GRAS

\titleformat{\section}[block]{\Large\bfseries}{\thesection \ }{0pt}{\Large \bfseries}
% Section : "1.1 TITRE" GRAND GRAS

\titleformat{\subsection}[block]{\large\bfseries}{\thesubsection \ }{0pt}{\large \bfseries}
% Sous-section : "1.1.1 TITRE" grand gras

%%%%%%%%%%%%%%%%%%%%%%%%%%%%%%%%%%%%%%%%%%%%%%%%%%%%%%%%%%%%%%%%%%%%%%%%%%%%%%%%
%%%%%%%%%%%%%%%%%%%%%%%%%%%%%%%%%%% CODES SOURCES %%%%%%%%%%%%%%%%%%%%%%%%%%%%%%%
%%%%%%%%%%%%%%%%%%%%%%%%%%%%%%%%%%%%%%%%%%%%%%%%%%%%%%%%%%%%%%%%%%%%%%%%%%%%%%%%
\usepackage{listings}								% Codes sources colorés

\lstset{
	language=Java,								% Langage DÉFAUT (changeable)
	numbers=left,								% Numéros lignes GAUCHE
	stepnumber=1,								% Numéro TOUTES les lignes
	numbersep=6pt,								% Distance numéros ↔ code
	numberstyle=\fontsize{9pt}{9pt}\ttfamily,	% Style numéros : petit monospace
	breaklines=true,							% Lignes longues → retour auto
	breakautoindent=true,						% Retour indenté
	postbreak=\space,							% Espace au retour
	tabsize=4,									% Tab = 4 espaces
	showspaces=false,							% Cache espaces invisibles
	showstringspaces=false,						% Cache espaces chaînes
	extendedchars=true,							% Caractères latins étendus
	captionpos=b,								% Légende EN BAS
	backgroundcolor=\color{lightGrey}, 			% Fond GRIS clair
	xleftmargin=0pt, xrightmargin=0pt,			% Pas marge latérale
	frame=leftline,								% Cadre GAUCHE seulement
	framerule=1.1pt,							% Épaisseur cadre 1.1pt
	frameround=ffff,							% Coins droits (pas arrondis)
	rulecolor=\color{lightBlue},				% Couleur cadre bleu clair
	framesep=1.5pt,								% Espace cadre ↔ code
	fillcolor=\color{lightGrey},				% Remplissage gris
	basicstyle=\ttfamily\footnotesize\color{darkGrey},	% Texte : petit gris foncé
	keywordstyle=\color{magenta}\bfseries,		% Mots-clés : magenta GRAS
	commentstyle=\color{lightBlue},				% Commentaires : bleu clair
	stringstyle=\color{darkBlue}				% Chaînes : bleu foncé
}

\lstloadlanguages{bash, C, C++, HTML, Java, PHP, Python, SQL}
% Support 8 langages avec coloration auto

% Personnalise titre liste codes
\renewcommand{\lstlistlistingname}{\ListOfListingsTitle}



% Répertoire des figures
\graphicspath{{../Images/}}











%	
%
%
%
% ============================================================================
%%%%% CORPS PRINCIPAL - Contenu scientifique
% ============================================================================
%%%%%%%%%%%%%%%%%%%%%%%%%%%%%%%%%%%%%%%%%%%%%%%%%%%%%%%%%%%%%%%%%%%%%%%%%%%%%%%%
%%%%%%%%%%%%%%%%%%%%%%%%%%%%%%%%%% Structure  %%%%%%%%%%%%%%%%%%%%%%%%%%%%%%%%%%
\begin{document}
	% ============================================================================
	% 1. CONFIGURATION INITIALE - Prépare les premières pages
	% ============================================================================

	\setlength{\headheight}{24.7pt}					% Hauteur minimale en-tête (requis Latexmk)

	\begin{titlepage}
	\includegraphics[width=4cm]{\companyLogo}
	\hfill
	\includegraphics[width=4cm]{\universityLogo}
	\centering
	
	{\large \ \par}
	{\scshape\LARGE \reportType \par}	
	\vspace{1.5cm}
	{\huge\bfseries \thesisTitle \par}
	\vspace{1.5cm}
	{\large \courseOfStudies \\ at \university \par}
	\vspace{1cm}
	{\large \byTitle \par}
	{\Large\itshape \authorName \par}
	\vfill
	{\large \thesisDate \par}
	\vfill
	\begin{tabbing}
		mmmmmmmmmmmmmmmmmmmmmmmmmm			\= \kill 			% define width
		\textbf{\timeOfProjectTitle}		\> \timeOfProject \\
		\textbf{\studentIdTitle}			\> \studentId \\
		\textbf{\companyTitle}				\> \company \\
		\textbf{\supervisorTitle}			\> \supervisorName \\
		\textbf{\secondSupervisorTitle}		\> \secondSupervisorName \\		
	\end{tabbing}
\end{titlepage}

% ############################################






									% Page titre, résumé, déclarations  					%%%%%%%%%%%%		FAIRE APPEL
	
	
	% 2 eme page
	%%%%% TABLE DES MATIÈRES - Préparation propre
	\setlength{\parskip}{0pt}						% Zéro espace après paragraphe
	\setcounter{tocdepth}{2}						% Niveaux : Chapitre + Section seulement
	\tableofcontents								% Génère table des matières
	\thispagestyle{empty}							% Page TOC sans numéro
	% 2 eme page
	
	
	% 3 eme page 
	\addtocounter{page}{6}							% +6 pages (compense frontmatter)
	\input{acronyms}								% Fichier abréviations									%%%%%%%%%%%%		FAIRE APPEL
	\addcontentsline{toc}{chapter}{\ListOfAbbreviationsTitle}	% Ajoute à TOC
	% 3 eme page
	

	%
	%
	%
	% ============================================================================
	%%%%% CORPS PRINCIPAL - Contenu scientifique
	% ============================================================================
	
	\pagenumbering{arabic}							% Pages 1, 2, 3... (mainmatter)   +++  arabic : Plus personne ne veut "i, ii, iii..." dans le contenu scientifique
	\setlength{\parskip}{8pt}						% 8pt après chaque paragraphe   +++  parskip 8pt : Paragraphes aérés (confort lecture)
	\renewcommand{\baselinestretch}{1.4}\normalsize	% Interlignes 1.4x + taille normale    
	% LE,RO : Gauche(paire) + Droite(impaire) = mise en page livre pro
	\fancyhead[LE,RO]{\thechapter  \ -- \ \nouppercase\leftmark}	% En-tête : "1 -- Chapitre"
	\fancyfoot[LE,RO]{\thepage}						% Pied : numéro page
	
	% ----------------------------------------------
\chapter{Structure}\label{ch:structure}
Beginning of the content core part  and i am trying the acronym \acs{cste} or i can also use \acf{cste} we can see the different between the two element

% ############################################
\section{Section}\label{sec:section1}

In enim justo, rhoncus ut, imperdiet a, venenatis vitae, justo. Nullam dictum felis eu pede mollis pretium. Integer tincidunt. Cras dapibus. Vivamus elementum semper nisi. Aenean vulputate eleifend tellus. Nulla consequat massa quis enim. Donec pede justo, fringilla vel, aliquet nec, vulputate eget, arcu.

\subsection{Subsection}

Phasellus viverra nulla ut metus varius laoreet. Quisque rutrum. Aenean imperdiet. Etiam ultricies nisi vel augue. Curabitur ullamcorper ultricies nisi. Nam eget dui. Etiam rhoncus. Maecenas tempus, tellus eget condimentum rhoncus, sem quam semper libero.

\subsubsection{Subsubsection}
Lorem ipsum dolor sit amet, consectetuer adipiscing elit. Aenean commodo ligula eget dolor. Aenean massa. Cum sociis natoque penatibus et magnis dis parturient montes, nascetur ridiculus mus. Donec quam felis, ultricies nec, pellentesque eu, pretium quis, sem. 

% ----------------------------------------------
\cleardoublepage
\chapter{Text Elements}

% ############################################
\section{New Commands}

This template provides some new commands:

You can add sources in image / listing captions (see \autoref{sec:images}), 
Andromeda University of Melmac
You can create definitions or theorems (see \autoref{sec:maths}).

You can format inline code in a paragraph (see \autoref{sec:code}).

% ############################################
\section{Acronyms and References}

A popular acronym is \acs{LOL}. \acf{IMHO} is another one. 
	
Citations can have page numbers \cite[p. 473]{smi:60}, \cite[pp. 359 - 360]{doe:16}, but they can be left out as well \cite{wai:99}, \cite{mus:16}, \cite{wri:81}, \cite{stu:89}.

You can also add footnotes.\footnote{It is displayed at the bottom of the page.}

\pagebreak

% ############################################
\section{Lists}

Unordered Lists:
\begin{itemize}
\item This is an unordered list. 
\item Item 2.
\item It has three items.
\end{itemize}

Ordered List:
\begin{enumerate}
\item This is an ordered list.
\item Item 2.
\item It has three items.
\end{enumerate}

Ordered List (alphabetical):
\begin{enumerate}[label=\Alph*.]
\item This is an ordered list.
\item Item 2.
\item It has three items.
\end{enumerate}

% ----------------------------------------------
\chapter{Figures}\label{ch:figures}

% ############################################
\section{Images}\label{sec:images}

\autoref{fig:image1} shows how to display images.

\begin{figure}[H]
	\centering
	\includegraphics[width=0.5\textwidth]{f1.jpg}
    \caption{Image}

	\label{fig:image1}
\end{figure}

\begin{figure}[H]
	\centering
	\includegraphics[width=0.5\textwidth]{f4.jpg}
    \caption{Image with Source}
    \captionsource{\cite{mus:16}}	
	\label{fig:image2}
\end{figure}

\begin{figure}[H]
	\centering
	\includegraphics[width=0.5\textwidth]{f7.jpg}
    \caption{Image with Source and Link}
    \captionsource[https://example.org]{J. Doe}	
	\label{fig:image3}
\end{figure}

% ############################################

% ----------------------------------------------

\chapter{Conclusion}\label{ch:conclusion}

% ############################################
\section{Mathematics}\label{sec:maths}

\subsection{Definitions}

\begin{define}	
A function $f: X \rightarrow Y$ is injective if and only if for all $x_1, x_2 \in X$, $x_1 \neq x_2$ implies $f(x_1) \neq f(x_2)$.
\end{define}

\subsection{Formulas}

Formulas can be included inline, e.g. $K_E^{CID}$ to $S$ or as an own block.
$$ otp = f_k(i, k), \qquad i = 1, ..., n $$

\pagebreak
% ############################################
\section{Code Listings}\label{sec:code}

Code can be displayed inline, e.g. \code{UPDATE} or \code{factorial(n)}.

Code listings can have numbered lines, captions, syntax highlighting, ... .

\begin{lstlisting}[caption={[Code Without Line Numbers] Code Without Line Numbers \\ \captionsource{\cite{mus:16}}}, label={lst:codeExample}, language=python, numbers=none]
 0  1  2  3  4  5  6  7  8  9 10 11 12 13 14 15 16 17 
 A  B  C  D  E  F  G  H  I  J  K  L  M  N  O  P  Q  R  
\end{lstlisting}

\begin{lstlisting}[caption={Code With Line Numbers}, label={lst:factorial}, language=Java]
public int factorial(int n) {
	if (n == 1) {
		return 1;
	} else {
		return n * factorial(n - 1) ;
	}
}
\end{lstlisting}
									% TOUT le contenu scientifique							%%%%%%%%%%%%		FAIRE APPEL
	\clearpage																					% Saut page avant biblio	
	%
	%
	%
	%
	% ============================================================================
	%%%%% BIBLIOGRAPHIE
	% ============================================================================
	
	\fancyhead[LE,RO]{\nouppercase\leftmark}		% En-tête simplifié
	\printbibliography[heading=bibintoc]			% Biblio + ajout automatique TOC
																			% Saut page IMPAIR annexes
	%
	%
	%
	%
	% ============================================================================
	%%%%% ANNEXES
	% ============================================================================
	
	\fancyhead[LE,RO]{A  \ -- \ \nouppercase\leftmark}	% En-tête : "A -- Annexe"
	\input{appendix}								% Contenu facultatifs si je veux seulement									%%%%%%%%%%%%		FAIRE APPEL
	
	
	
	
	
	
	
	
\end{document}






% All the file important to run this code are: 
%		fried-addons.tex for the first page shape design 
%		acronyms.tex for the all about abbreviation that i use in the documents 
%		content.tex for the main part or develpment
%		appendix.tex for the more explanation about the main development but separatly