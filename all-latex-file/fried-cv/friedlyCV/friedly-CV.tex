

%============================================================================%
%
%	DOCUMENT DEFINITION
%
%============================================================================%

%we use article class because we want to fully customize the page and don't use a cv template
\documentclass[10pt,A4]{article}	


%----------------------------------------------------------------------------------------
%	ENCODING
%----------------------------------------------------------------------------------------

% we use utf8 since we want to build from any machine
\usepackage[utf8]{inputenc}		

%----------------------------------------------------------------------------------------
%	LOGIC
%----------------------------------------------------------------------------------------

% provides \isempty test
\usepackage{xstring, xifthen}

%----------------------------------------------------------------------------------------
%	FONT BASICS
%----------------------------------------------------------------------------------------

% some tex-live fonts - choose your own

%\usepackage[defaultsans]{droidsans}
%\usepackage[default]{comfortaa}
%\usepackage{cmbright}
%\usepackage[default]{raleway}
%\usepackage{fetamont}
%\usepackage[default]{gillius}
%\usepackage[light,math]{iwona}
%\usepackage[thin]{roboto} 

% set font default
\renewcommand*\familydefault{\sfdefault} 	
\usepackage[T1]{fontenc}

% more font size definitions
\usepackage{moresize}

%----------------------------------------------------------------------------------------
%	FONT AWESOME ICONS
%---------------------------------------------------------------------------------------- 

% include the fontawesome icon set
\usepackage{fontawesome}

% use to vertically center content
% credits to: http://tex.stackexchange.com/questions/7219/how-to-vertically-center-two-images-next-to-each-other
\newcommand{\vcenteredinclude}[1]{\begingroup
\setbox0=\hbox{\includegraphics{#1}}%
\parbox{\wd0}{\box0}\endgroup}

% use to vertically center content
% credits to: http://tex.stackexchange.com/questions/7219/how-to-vertically-center-two-images-next-to-each-other
\newcommand*{\vcenteredhbox}[1]{\begingroup
\setbox0=\hbox{#1}\parbox{\wd0}{\box0}\endgroup}

% icon shortcut
\newcommand{\icon}[3] { 							
	\makebox(#2, #2){\textcolor{maincol}{\csname fa#1\endcsname}}
}	

% icon with text shortcut
\newcommand{\icontext}[4]{ 						
	\vcenteredhbox{\icon{#1}{#2}{#3}}  \hspace{2pt}  \parbox{0.9\mpwidth}{\textcolor{#4}{#3}}
}

% icon with website url
\newcommand{\iconhref}[5]{ 						
    \vcenteredhbox{\icon{#1}{#2}{#5}}  \hspace{2pt} \href{#4}{\textcolor{#5}{#3}}
}

% icon with email link
\newcommand{\iconemail}[5]{ 						
    \vcenteredhbox{\icon{#1}{#2}{#5}}  \hspace{2pt} \href{mailto:#4}{\textcolor{#5}{#3}}
}

%----------------------------------------------------------------------------------------
%	PAGE LAYOUT  DEFINITIONS
%----------------------------------------------------------------------------------------

% page outer frames (debug-only)
% \usepackage{showframe}		

% we use paracol to display breakable two columns
\usepackage{paracol}

% define page styles using geometry
\usepackage[a4paper]{geometry}

% remove all possible margins
\geometry{top=1cm, bottom=1cm, left=1cm, right=1cm}

\usepackage{fancyhdr}
\pagestyle{empty}

% space between header and content
\setlength{\headheight}{0pt}

% indentation is zero
\setlength{\parindent}{0mm}

%----------------------------------------------------------------------------------------
%	TABLE /ARRAY DEFINITIONS
%---------------------------------------------------------------------------------------- 

% extended aligning of tabular cells
\usepackage{array}

% custom column right-align with fixed width
% use like p{size} but via x{size}
\newcolumntype{x}[1]{%
>{\raggedleft\hspace{0pt}}p{#1}}%


%----------------------------------------------------------------------------------------
%	GRAPHICS DEFINITIONS
%---------------------------------------------------------------------------------------- 

%for header image
\usepackage{graphicx}

% use this for floating figures
% \usepackage{wrapfig}
% \usepackage{float}
% \floatstyle{boxed} 
% \restylefloat{figure}

%for drawing graphics		
\usepackage{tikz}				
\usetikzlibrary{shapes, backgrounds,mindmap, trees}

%----------------------------------------------------------------------------------------
%	Color DEFINITIONS
%---------------------------------------------------------------------------------------- 
\usepackage{transparent}
\usepackage{color}

% primary color
\definecolor{maincol}{RGB}{ 225, 0, 0 }

% accent color, secondary
% \definecolor{accentcol}{RGB}{ 250, 150, 10 }

% dark color
\definecolor{darkcol}{RGB}{70, 70, 70}  % ← GRIS FONCÉ ACTUEL


% light color
\definecolor{lightcol}{RGB}{245,245,245}


% Package for links, must be the last package used
\usepackage[hidelinks]{hyperref}

% returns minipage width minus two times \fboxsep
% to keep padding included in width calculations
% can also be used for other boxes / environments
\newcommand{\mpwidth}{\linewidth-\fboxsep-\fboxsep}
	


%============================================================================%
%
%	CV COMMANDS
%
%============================================================================%

%----------------------------------------------------------------------------------------
%	 CV LIST
%----------------------------------------------------------------------------------------

% renders a standard latex list but abstracts away the environment definition (begin/end)
\newcommand{\cvlist}[1]{%
	\begin{itemize}
		\setlength{\itemsep}{1pt}   % espace entre puces
		\setlength{\topsep}{2pt}    % espace avant/après la liste
		#1
	\end{itemize}
}
%----------------------------------------------------------------------------------------
%	 CV TEXT
%----------------------------------------------------------------------------------------

% base class to wrap any text based stuff here. Renders like a paragraph.
% Allows complex commands to be passed, too.
% param 1: *any
\newcommand{\cvtext}[1] {
	\begin{tabular*}{1\mpwidth}{p{0.98\mpwidth}}
		\parbox{1\mpwidth}{#1}
	\end{tabular*}
}

%----------------------------------------------------------------------------------------
%	CV SECTION
%----------------------------------------------------------------------------------------

% Renders a a CV section headline with a nice underline in main color.
% param 1: section title
\newcommand{\cvsection}[1] {
	\vspace{8pt}
	\cvtext{
		\textbf{\Large{\textcolor{darkcol}{\uppercase{#1}}}}\\[-2pt]
		%\textcolor{maincol}{ \rule{0.1\textwidth}{0.5pt} } \\
	}
}

%----------------------------------------------------------------------------------------
%	META SKILL
%----------------------------------------------------------------------------------------


\newcommand{\cvskill}[3] {
	\begin{tabular*}{1\mpwidth}{p{0.72\mpwidth}  r}
 		\textcolor{black}{\textbf{#1}} & \textcolor{maincol}{#2}\\
	\end{tabular*}%
	
	\hspace{2pt}
	\begin{tikzpicture}[scale=1,rounded corners=2pt,very thin]
		\fill [lightcol] (0,0) rectangle (1\mpwidth, 0.12);
		\fill [maincol] (0,0) rectangle (#3\mpwidth, 0.12);
  	\end{tikzpicture}%
}


%----------------------------------------------------------------------------------------
%	 CV EVENT
%----------------------------------------------------------------------------------------

\newcommand{\cvevent}[7] {
	
	\parbox{\mpwidth}{
		\begin{tabular*}{1\mpwidth}{p{0.72\mpwidth} r}
			\textcolor{black}{\textbf{#2}} &
			\colorbox{maincol}{\makebox[0.30\mpwidth]{\textcolor{white}{#1}}} \\
			\textcolor{maincol}{\textbf{#3}} & \\
		\end{tabular*}\\[4pt]
	}
	
	
	\ifthenelse{\isempty{#4}}{}{
		 \cvtext{#4}\\[-4pt]
	}
	
	
	\ifthenelse{\isempty{#5}}{}{
		{#5}
	}
	
	
}


%----------------------------------------------------------------------------------------
%	 CV META EVENT
%----------------------------------------------------------------------------------------

\newcommand{\cvmetaevent}[4] {
	\textcolor{maincol} {\cvtext{\textbf{\begin{flushleft}#1\end{flushleft}}}}

	\ifthenelse{\isempty{#2}}{}{
	\textcolor{darkcol} {\cvtext{\textbf{#2}} }
	}

	\ifthenelse{\isempty{#2}}{}{
		\textcolor{darkcol} {\cvtext{\textbf{#2}} }
	}

	\cvtext{#4}\\[14pt]
}




%============================================================================%
%
%
%
%	DOCUMENT CONTENT
%
%
%
%============================================================================%
\begin{document}
	\columnratio{0.31}
	\setlength{\columnsep}{1.5em}
	\setlength{\columnseprule}{3pt}
	\colseprulecolor{lightcol}
	
	\begin{paracol}{2}
		
		
		\begin{leftcolumn}
			%---------------------------------------------------------------------------------------
			%	META IMAGE
			%----------------------------------------------------------------------------------------
			\includegraphics[width=\linewidth]{Untitled.jpg}	
			
			%---------------------------------------------------------------------------------------
			%	META SKILLS
			%---------------------------------------------------------------------------------------
			\vspace{8pt}
			\cvsection{CONTACT}
			
				\icontext{MapMarker}{12}{3 Rue George Bernanos}{black}\\[10pt]
				\icontext{MobilePhone}{12}{+33 7 67 33 92 90}{black}\\[10pt]
				\iconemail{Envelope}{12}{friedlysedjro@gmail.com}{contact@example.com}{black}\\[10pt]
				\iconhref{Github}{12}{github.com/Friedsed}{https://github.com/Friedsed}{black}\\[10pt]
				\iconhref{Linkedin}{12}{linkedin.com/in/friedly-w}{https://www.linkedin.com/in/friedly-w/}{black}
			
				
			
			\vspace*{8pt}
			
			\cvsection{COMPÉTENCES}
			
				\cvskill{Python/ Matlab} {} {0.7} \\[-2pt]
				
				\cvskill{Ansys/Abaqus} {} {0.6} \\[-2pt]
				
				\cvskill{3D Experience} {} {0.5} \\[-2pt]
				
				\cvskill{Html/CSS/JS} {} {0.5} \\[-2pt]
				
				\cvskill{C++} {} {0.4} \\[-2pt]
			
			\vspace*{8pt}
			\cvsection{LANGUES}
			
				\cvskill{Français} {} {0.95} \\[-2pt]
				
				\cvskill{Anglais} {} {0.8} \\[-2pt]
				
				\cvskill{Allemand} {} {0.2} \\[-2pt]
				
			\vspace*{8pt}	
			\cvsection{CENTRE D'INTERET}
				
				\cvlist{
					\item \textbf{Sport} : course à pied, basketball
					\item \textbf{Voyage} : Angleterre, Allemangne, Espagne
				
				
				}
				

			
			
		
		
			
			
		\end{leftcolumn}
		
		
		
		
		\begin{rightcolumn}
			
			
			%---------------------------------------------------------------------------------------
			%	TITLE  HEADER
			%----------------------------------------------------------------------------------------
			\fcolorbox{white}{darkcol}{\begin{minipage}[c][3cm][c]{1\mpwidth}
				\begin {center}
					\Huge{ \textbf{ \textcolor{white}{ \uppercase{ Friedly WOLI } } } } \\[-24pt]
					\textcolor{white}{ \rule{0.1\textwidth}{1.25pt} } \\[4pt]
					\large{ \textcolor{white} {Recherche d'alternance en simulation numérique  } }
				\end {center}
			\end{minipage}} \\[14pt]
			\vspace{-12pt}
			
			%---------------------------------------------------------------------------------------
			%	PROFILE
			%----------------------------------------------------------------------------------------
			%
			\cvsection{PROFILE}
			
				\cvtext{Étudiant en L3 Mécanique Énergétique, je suis à la recherche d’un contrat d’alternance débutant en Septembre 2026 dans le domaine de la mécanique et de la simulation numérique. Mon objectif est de m'investir au sein d’une équipe sur le long terme pour contribuer à des projets d'innovation complexes
				}
			
			%---------------------------------------------------------------------------------------
			%	Formation
			%----------------------------------------------------------------------------------------
			\vspace{12pt}
			\cvsection{FORMATION}
			
				\cvevent
				{Sept 2024 - Juin 2025}
				{Université de Toulouse}
				{L3 Mécanique énergétique}
				{}
				{\cvlist{
						\item Mécanique analytique, Mécanique des milieux continus, Mécanique des fluides, Mécanique du vol
						\item  Résistance des matériaux,Transfert thermique, Programmation Python, Calcul scientifique
				}}
				\\\\
				\cvevent
				{Sept 2024 - Juin 2025}
				{INSA Toulouse}        % pas d'espace avant/après
				{Génie mécanique}
				{}
				{\cvlist{
						\item Conception mécanique, CAO avec Creo et 3D Experience, Métrologie, Asservisement
						\item Membre du club robot de l'INSA Toulouse
				}}
				\\\\
				\cvevent
				{Sept 2022 - Juin 2024}
				{Université de Lorraine}     % pas d'espace avant/après
				{Cycle Universitaire Préparatoire aux Grandes Écoles }
				{}
				{\cvlist{
						 \item Analyse différentielle, Algèbre multilinéaire, Probabilités, Géométrie, C++
						\item Astronomie, Électromagnétisme, Optique ondulatoire, Analyse numérique
				}}
				
		
			
			
			
			%---------------------------------------------------------------------------------------
			%	WORK EXPERIENCE
			%----------------------------------------------------------------------------------------

			\cvsection{EXPÉRIENCE PROFESSIONNELLE}
			
				\cvevent
				{Jan 2025}
				{ISAE Supaero}
				{Toulouse Racing Division: Powertrain}
				{}
				{\cvlist{
						\item Projet de conception d’un véhicule électrique
						\item Codage à l’aide de Matlab et de Simulink du fonctionnement global du véhicule
						\item Présentation du projet à Silverstone en juillet 2025 devant des jurys internationaux, en tant que rapporteur de groupe
				}}
				\\\\
			\cvevent
				{Sept 2024 - Juin 2025}
				{Projet personnel}
				{Développeur}
				{}
				{\cvlist{
						\item Développement d'un site Web interactif pour la résolution de problèmes mathématiques
						\item Conception avec les langages de programmation HTML, CSS et JavaScript
				}}
			\\\\
			\cvevent
				{Sept 2022 - Mars 2023}
				{Mairie Nancy}
				{Tutorat}
				{}
				{\cvlist{
						\item Organisation et animation de séances de tutorat en mathématiques et en physique pour des lycéens
						\item Accompagnement individualisé d'élèves (collège/lycée) en maths/physique pour consolider les bases et préparer les examens
				}}
			
				
				
				
				  
				
			% hotfixes to create fake-space to ensure the whole height is used
		
		\end{rightcolumn}
	\end{paracol}
\end{document}

